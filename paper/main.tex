% 10+2 Due Fri 28 Aug anywhere-on-earth
%
% ICSE 2019 Technical Track submission must not exceed 10 pages,
% including all text, figures, tables, and appendices; two additional pages
% containing only references are permitted.
%
% ICSE 2019 Technical Track will employ a double-blind review process.
% Thus, no submission may reveal its authors’ identities.
%
% https://2019.icse-conferences.org/track/icse-2019-Technical-Papers#Call-for-Papers

\documentclass[conference]{IEEEtran}
\IEEEoverridecommandlockouts
\usepackage{cite}
\usepackage{amsmath,amssymb,amsfonts}
\usepackage{algorithmic}
\usepackage{graphicx}
\usepackage{booktabs}
\usepackage{textcomp}
\usepackage{xcolor}
\usepackage{listings}
\usepackage{myref}
\usepackage{pgfplots}
\def\BibTeX{{\rm B\kern-.05em{\sc i\kern-.025em b}\kern-.08em
    T\kern-.1667em\lower.7ex\hbox{E}\kern-.125emX}}

\newcommand\lt[1]{{\lstinline|#1|}}
\lstset{language=python}
\definecolor{dkgreen}{rgb}{0,0.5,0}
\definecolor{dkred}{rgb}{0.5,0,0}
\definecolor{gray}{rgb}{0.5,0.5,0.5}
\lstset{basicstyle=\ttfamily\bfseries\footnotesize,
  morekeywords={virtualinvoke},
  keywordstyle=\color{blue},
  ndkeywordstyle=\color{red},
  commentstyle=\color{dkred},
  stringstyle=\color{dkgreen},
  numbers=left,
  numberstyle=\ttfamily\footnotesize\color{gray},
  stepnumber=1,
  numbersep=10pt,
  backgroundcolor=\color{white},
  tabsize=4,
  showspaces=false,
  showstringspaces=false,
  xleftmargin=.23in
}


\begin{document}

%
\title{Blaming the Typeless: \\ Scalable, Human-Centric Python Localization}

\iffalse

\author{\IEEEauthorblockN{Benjamin Cosman}
\IEEEauthorblockA{\textit{UC San Diego}\\
blcosman@eng.ecsd.edu}
\and
\IEEEauthorblockN{Leon Medvinsky}
\IEEEauthorblockA{\textit{UC San Diego}\\
lmedvinsky@eng.ecsd.edu}
\and
\IEEEauthorblockN{Ranjit Jhala}
\IEEEauthorblockA{\textit{UC San Diego}\\
jhala@cs.ecsd.edu}
\and
\IEEEauthorblockN{Westley Weimer}
\IEEEauthorblockA{\textit{University of Michigan}\\
weimerw@umich.edu}
}

\fi

\author{\IEEEauthorblockN{\emph{submitted for double-blind review}}}

\maketitle

\begin{abstract}
Abstract
\end{abstract}

\section{Introduction}

Dynamically-typed languages are becoming increasingly common for rapid
application development, full-scale software engineering, and
pedagogy (e.g.,~\cite{FIXME,FIXME}). Languages like Python, Ruby and Lua
often feature a gentle learning curve and attractive facilities, from
familiar garbage collection and regular expressions to more expressive and
exotic list comprehensions and higher-order function blocks. One side
effect of popularity with novices and students, however, is that many
programmers of such languages may lack expertise when presented with
dynamic type errors, and may thus struggle to interpret and localize
them~\cite{FIXME,FIXME}. We propose a machine learning approach based on
static, dynamic and contextual information that effectively localizes
beginner type errors.

The relative costs and benefits of static vs. dynamic error detection are
well-studied. Despite this, however, many currently-available techniques
are ill-suited to address the problem of localization for type errors in
dynamic languages. First, relying on the extant language interpreter
requires test inputs~\cite{FIXME}, may involve error messages that are
difficult for novices to understand~\cite{FIXME}, and may misleadingly
implicate the symptom rather than the cause~\cite{FIXME}.  Second, a
significant amount of development is carried out via IDE or web interfaces,
which require tools that operate on code fragments~\cite{Guo2013-vu}.
Third, attempts to retrofit static type systems into dynamic languages,
such as TypeScript for JavaScript~\cite{FIXME} and FIXME for
Python~\cite{FIXME} are promising but not yet as widely
deployed~\cite{FIXME}. Finally, research tools and algorithms for fault
localization either require rich static types (e.g., as in
Mycroft~\cite{FIXME} or Sherrloc~\cite{FIXME}), require test cases (e.g.,
as in Tarantula~\cite{FIXME}), or produce voluminous ranked output lists
that have been found to be unhelpful in general and less useful to novices
in particular~\cite[Sec.~5.1]{orso-parnin}. 

We focus on Python as an indicative, popular dynamically-typed language.
We desire a fault localization algorithm for Python type errors that
will agree with human intuition (\emph{accuracy}), will apply to
off-the-shelf, unannotated beginner-written program fragments
(\emph{generality}), and will scalably apply to industrial-strength
deployments (\emph{scalability}). Our key insights are (1) that we can leverage
modern supervised machine learning for accuracy and an agreement with human
norms; (2) that we can use static, dynamic, contextual and slice-based
features to generalize to off-the-shelf beginner programs; and (3) that we
can use large beginner datasets, such as those from
PythonTutor.com~\cite{Guo2012-vu}, to assess scalability.

We propose an algorithm that takes an input a Python program fragment that 
produces an uncaught runtime exception such as a type error. Via machine
learning over a carefully-selected set of program-relevant features, we
produce the expression most likely implicated in the exception according to
our learned model. We evaluate on an entire year of buggy Python 3 program
executions on PythonTutor.com --- over 270,000 instances --- measuring
accuracy with respect to the human actions as the ground truth and
comparing against the vanilla Python interpreter as a baseline. 

The contributions of this paper are as follows:
\begin{enumerate}

\item We present an algorithm for accurately localizing dynamic type errors
in Python. We find that FIXME. 

\item We describe and evaluate static, dynamic, contextual and slice-based
features for generalizing machine-learning fault localization to
beginner-written Python fragments. We find that FIXME.

\item We analyze the results of a thorough evaluation on over 270,000 real
user instances, demonstrating the scalability of our algorithm. 

\end{enumerate} 

\section{Motivating Example}

\begin{figure}
\begin{lstlisting}
year = int(time.strftime("%Y"))
age = input("Enter your age")
print("You will be twice as old in:")
print(year + age)
\end{lstlisting}
\caption{
\label{fig-motex}
A program with multiple fault localizations (e.g.,
lines 1 and 2).}
\end{figure}

Consider the program in \figref{motex}, adapted from our dataset of
human submissions. The program attempts to carry out some simple arithmetic
based on a given number and the current year. When executed, however, the
program raises an exception on line 4, related to the addition of an
integer variable to a string variable.

One reasonable fix (and the one that the programmer actually used in this
case) is to add \lt{int()} cast on line 2. However, another possible
``fix'' would be to \textit{remove} the \lt{int()} cast from line 1, in
which case both variables hold string values at runtime, and the addition
on line 4 would be interpreted as string concatenation. While the second
fix creates a well-typed program, it is less likely to correspond to
developer intent, and is thus less likely to be helpful as a debugging aid.

This simple example highlights a key fault localization challenge in such
dynamically-typed settings: a runtime type error alone may not contain
enough information to pinpoint the human-preferred localization from among
many valid-but-less-helpful localizations. Our insight is that the relevant
information is not in the types (e.g., string or integer) alone but in the
context in which the values are used.  We propose an approach based on
machine learning over a combination of static and dynamic contextual
features.

% TODO: would also be nice if the example showed that the place Python crashes
% may not be a place that should be fixed?

\section{Overview and Approach}

We present an algorithm for accurately localizing faults~\cite{tarantula} in
dynamically-typed Python programs that exhibit non-trivial uncaught runtime
exceptions. We do not consider syntax errors or references to undefined
variables. Our algorithm uses machine learning models based on static and
dynamic features to implicate suspicious expressions. Since studies have
found that voluminous fault localization output is not useful to
developers~\cite{orso-parnin,orso-parnin2015}, we focus on producing
Top-1 and Top-3 rankings.

Our algorithm first extracts static and dynamic features from Python
program (Section~\ref{sec-features}). Next, using a labeled training
corpus, we learn a machine learning model over those features
(Section~\ref{sec-model}). Once the model has been learned, we localize
faults in new Python programs by extracting their features and applying the
model.

\subsection{Feature Extraction}
\label{sec-features}

Our key insight is that FIXME-INSIGHT. Are a result, we make use of
static, dynamic and contextual information. Static features, such as FIXME,
are effective at capturing FIXME-INFORMATION. By contrast, dynamic
features, such as FIXME, support the modeling of FIXME. Finally, we also
include contextual information, such as FIXME, which allows our algorithm
to accurately handle FIXME. We consider the empirical justification of
these features in Section~\ref{sec-eval}.

We calculate features for each statement and expression in the program.

\begin{itemize}

\item \emph{Static / syntactic features}
\begin{enumerate}
    \item What kind of statement/expression it is, e.g. assignment / return
     / import (for statements) or variable / literal / application (expressions).
    \item Size (the number of AST nodes in the subtree rooted at this node)
\end{enumerate}

\item \emph{Dynamic features}
We run each program through an instrumented interpreter (i.e. a modified PythonTutor
backend, which itself is based on the BDB debugging library). This lets us
compute the following features:
\begin{enumerate}
    \item What type(s) does the expression have at runtime? (The values of this feature
    can be a type like int, or the special values "unknown" if the expression is
    never successfully evaluated and "multiple" if it changes type.)
    \item Is it part of the error slice?
    \item Is it the node at which the program actually crashes?
\end{enumerate}
(Since this interpreter works at the granularity of lines and we want to work with
expressions as well, we first convert each program to ANF.)

We compute the error slice by running the program and building a dependency
graph where node A depends on node B if either B defines a variable that A uses,
or B affects control flow allowing A to run.

We also check what uncaught exception is thrown by the program (TypeError,
KeyError, etc), and add that as a constant feature to all vectors derived from that program.

\item \emph{Contextual features} After all other features are computed so we have a
vector $v_e$ representing expression $e$, we set $v_e^{+context} = v_e \circ
v_p \circ v_{c_1} \circ v_{c_2} \circ v_{c_3}$ where $p$ is the expression's parent
in the AST and $c_i$ are its children. This allows us to recover some of the
program structure that would otherwise be completely lost when we convert the
structured program into an unstructured bag of feature vectors.

\item \emph{Label} Each feature vector is labeled by whether that expression/statement
changed between the crashing and fixed versions. (This is our proxy for whether
that node is to blame for the crash).

\end{itemize}

Each vector is then one-hot encoded (e.g. feature "Type : Int" becomes set
of features "Type-Int : 1", "Type-Bool : 0", "Type-Str : 0", etc.) for use with
standard machine learning tools.

\subsection{Machine Learning and Modeling}
\label{sec-model}

The models we train (inc. decision trees/forests,
MLP, GBM) can then rank expressions of a test program in order of how likely they
are to be the source of the bug, so we can compute Top-k accuracies: how often
is the true program change located in the model's first k outputs.

\section{Approach, in more detail}

We follow the overall strategy of \cite{learning-to-blame} and represent each
buggy program by a Bag of Abstracted Terms. In our setting, a `term' means either
a statment or an expression.

For each term in the program we compute static, dynamic,
and contextual features. While it is easy to read some features off of the surface
syntax, we expect that some features which are only accessible dynamically will also
be useful, for example the type of each expression. Since programs are highly
structured and expressions gain meaning in relation to their surroundings,
we also expect contextual features to be useful.

\subsection{Static features}

\emph{Syntactic form} We expect certain kinds of terms to be more prone to bugs
than others - for example, students might have more trouble with loop conditions
than with simple assignments. Thus the first feature we consider is what kind of node the term
is in the AST. For statements this has values like Return or Import; for expressions
Variable or Application.

\emph{Size} We also count the number of descendent nodes in the subtree rooted at that node.

\subsection{Dynamic features}

We first convert each buggy program to an equivalent program that is closer
to ANF, then run it through the PythonTutor backend. This backend is based on the
Python debigging library BDB; it returns to us a trace of the program along with
the state of the heap at each execution step. We use this information to compute
the following features for each term:

\emph{Type} An expression's type should be a very useful feature. Not only are
incompatible types a sign of an impending crash, but some types may be inherently
suspect whenever they appear, e.g. there are very few reasons to have a variable
of type NoneType. Our possible values for the Type feature include all the basic Python
types (such as int and tuple) as well as a few special values:
\begin{itemize}
    \item IsStatement - Given to all statements, since our terms include both
    statments and expressions but only expressions actually have types.
    \item Unknown - Given to expressions that are never evaluated in the dynamic
    trace.
    \item Multiple - Given to expressions which are evaluated multiple times in
    a trace and do not always have the same type.
\end{itemize}

\emph{Slice}
\begin{figure}
\begin{lstlisting}
x = input() # The user inputs 0
if x != 0:
  x = 1
print("One divided by your number is: %d" % (1 / x))
\end{lstlisting}
\caption{In this program, the programmer may have intended the placeholder
  assignment \texttt{x = 1} to be used in case the user input 0, and used the
  incorrect condition \texttt{x != 0}, rather than \texttt{x == 0}. However, the
  condition expression will not be included in the slice (which would include
  lines 1 and 4), because the presence of
  the conditional does not affect the behavior of the program.
}
\label{slice-downside-example}
\end{figure}

\begin{figure}
\begin{lstlisting}
while true:
  x = input()
  if x != 0:
    break
  print("One divided by your number is: %d" % (1 / x))
\end{lstlisting}
\caption{
  This program's mistake is the same as before,
  but instead they try to handle the
  bad input by breaking out of the input loop. When computing the slice, our
  method would record a control dependency between the condition on line 3 and
  the print statement on line 5, placing the bad condition in the program slice.
}
\label{early-break}
\end{figure}

The goal of this feature, roughly equivalent in purpose to the type error slice in
\cite {learning-to-blame}, is to help eliminate terms that cannot be the source of
the crash. We compute a dynamic program slice \cite{KOREL1988155}, the set of
terms that contributed at runtime to the observed exception, and turn this
feature on for terms in the slice. We follow the basic approach of
\cite{KOREL1988155, KOREL1990187}, building a graph of data and control
dependencies. We then traverse the graph backwards, starting at the execution
step where the error occurred, to collect the set of terms that the excepting
line transitively depended on. This excludes lines that could not have caused the
exception, such as lines that never ran, or lines that ran but had no connection
to the step where the exception happened.

A downside of using the dynamic slice, in the case of control dependencies, is
that it excludes terms whose presence does not affect whether the exception
happened, but which could have had an effect if the term was different. In
\figref{slice-downside-example}, the conditional statement is never run, so the
removal of the conditional would not have any effect on the program's behavior,
and it would not be included in the program slice. The dynamic slice misses the
incorrect test condition because the mistake itself caused the conditional
statement and its body to become irrelevant.

To overcome this issues in the ``early return'' or ``early break'' case
(\figref{early-break}), we check if a break, return, or other statement for
breaking out of structured control flow is present inside of a conditional
statement. We then add dependencies in the execution trace's dependency graph
between the enclosing conditional and the statements that would have been skipped
by the break or return. This is insufficient to overcome all related problems,
however, so unlike in prior work, we do not treat membership in the slice
as a hard constraint.

\emph{Crash location} As discussed later in the evaluation, the precise term that
causes Python to crash is in fact the one the user decides to change about 30\% of
the time, so we use it as a feature (and expect it to be a high-signal one, too.)

\emph{Exception type} The type of error thrown by the program is useful context for
finding the error. For example, if asked how likely a random `0` is of being a bug, you'd give
a higher answer if you knew that the program eventually crashes with a division by
zero exception. Thus we attach the exception type as a feature to \emph{every}
vector derived from that program.

\subsection{Contextual features}

The approach of representing a program by a Bag of Abstracted Terms, if implemented
naively, runs into the problem that the terms lose their context in the program
as a whole. For example, the features for the term `0` include its type (int) and
syntactic description (int literal), but there is no way that's enough to decide
if it's innocuous, like the 0 in `x = 0`, or clearly the source of a bug, like the
0 in `x / 0`. To restore some of this context, we introduce contextual features.
This involves duplicating all our other features four times - once for the parent
and once each for a term's first three children in the AST. So since `x / 0` has
the feature SyntacticForm = DivisionOp, the `0` of `x / 0` has the feature
ParentSyntacticForm = DivisionOp. Similarly, since `0` has SyntacticForm = IntLiteral,
`x / 0` has Child1SyntacticForm = IntLiteral. In order to make all feature vectors
the same length, nodes that have no parent or fewer than three children have the
relevant features filled out with a special NotApplicable value.

\subsection{Label}

The label we would ideally like to use would be whether or not a term is the
true source of the crash; that is, whether the correct way to fix the crash is
to edit this term. However, manually labeling terms in such a way would be
prohibitive on such a large dataset. And the whole point of this work is that
there does not yet exist a reliable way of pinpointing bugs automatically. Thus
we are forced to use a proxy for our desired label as follows. Many PythonTutor
users use the site in an iterative fashion: they start out by writing
a program that crashes, and then edit it until it no longer crashes. Thus for our
dataset, we consider only the crashing programs for which the same user later
ran a program that did not crash. We then run a tree-diff \cite{tree-diff}
between the program that crashes and the one that does not. Any terms in the
crashing program that appear in the diff are labeled as bugs. This is of course
not a perfect proxy for the label we want; see Threats to Validity.

\subsection{Machine learning}

\emph{Pairing programs} For each PythonTutor user, we look at the sequence of
programs they attempted to run. Each program that crashes is paired with the next
program by that user which does not crash. Many programs do not make it to the
end of our data pipeline; here are the most common reasons:
\begin{itemize}
    \item The user submitted only programs that crash or only programs that do
    not crash, so we can't create any pairs for that user.
    \item The program is out of scope for PythonTutor. For
    example, the program imports forbidden libraries (like os), or the program
    runs in over 1000 execution steps.
    \item The proportion of the program that has changed between the buggy and
    fixed versions is more than a standard deviation above the mean.
\end{itemize}
We end up with \textasciitilde270,000 usable pairs of programs.

\emph{Models} The following descriptions of Logistic Regression, Decision Trees,
and MLP will have to be redone as they are copied from Learning to Blame.

\emph{Logistic Regression}
The simplest classifier we investigate is logistic regression:
a linear model where the goal is to learn a set of weights $W$
that describe the following model for predicting a label
$b$ from a feature vector $v$:
%
\[ \Pr(b = 1 | v) = \frac{1}{1 + e^{-W^{\top} v}} \]
%
The weights $W$ are learnt from training data, and the value of
$\Pr(b | v)$ naturally leads to a confidence score $\mathcal{C}$.
%
Logistic regression is a widely used classification algorithm, preferred
for its simplicity, ease of generalization, and interpretability.
%
Its main limitation is that the prediction rule is constrained to be a
linear combination of the features, and hence relatively simple.
%
While this can be somewhat mitigated by adding higher order (quadratic
or cubic) features, this often requires substantial domain knowledge.

\emph{Decision Trees}
Decision tree algorithms learn a tree of binary predicates over the
features, recursively partitioning the input space until a final
classification can be made.
%
Each node in the tree contains a single predicate of the form
$v_j \leq t$ for some feature $v_j$ and threshold $t$, which determines
whether a given input should proceed down the left or right subtree.
%
Each leaf is labeled with a prediction and the fraction of
correctly-labeled training samples that would reach it; the latter
quantity can be interpreted as the decision tree's confidence in its
prediction.
%
This leads to a prediction rule that can be quite expressive depending
on the data used to build it.

Training a decision tree entails finding both a set of good partitioning
predicates and a good ordering of the predicates based on data.
%
This is usually done in a top-down greedy manner, and there are several
standard training algorithms such as C4.5 \cite{Quinlan1993-de} and
CART \cite{Breiman1984-qy}.

Another advantage of decision trees is their ease of interpretation ---
the decision rule is a white-box model that can be readily described to
a human, especially when the tree is small.
%
However, the main limitation is that these trees often do not generalize
well, though this can be somewhat mitigated by \emph{pruning} the tree.

\emph{Neural Networks}
%
The last (and most complex) model we use is a type of neural network
called a \emph{multi-layer perceptron} (see \cite{Nielsen2015-pu} for
an introduction to neural networks).
%
A multi-layer perceptron can be represented as a directed acyclic
graph whose nodes are arranged in layers that are fully connected by
weighted edges.
%
The first layer corresponds to the input features, and the final to the
output.
%
%The output of a node $v$ in an internal layer is given by:
The output of an internal node $v$ is
%
\[ h_v = g(\sum_{j \in N(v)} W_{jv} h_j ) \]
%
where $N(v)$ is the set of nodes in the previous layer that are adjacent
to $v$, $W_{jv}$ is the weight of the $(j, v)$ edge and $h_j$ is the
output of node $j$ in the previous layer.
%
Finally $g$ is a non-linear function, called the activation function,
which in recent work is commonly chosen to be the \emph{rectified linear
  unit} (ReLU), defined as $g(x) = \mathsf{max}(0,x)$
\cite{Nair2010-xg}.
%
The number of layers, the number of neurons per layer, and the
connections between layers constitute the \emph{architecture} of a
neural network.
%
In this work, we use relatively simple neural networks which have an
input layer, a single hidden layer and an output layer.

A major advantage of neural networks is their ability to discover
interesting combinations of features through non-linearity, which
significantly reduces the need for manual feature engineering, and
allows high expressivity.
%
On the other hand, this makes the networks particularly difficult to
interpret and also difficult to generalize unless vast amounts of
training data are available.

\emph{Training methodology} At this point we have feature vectors describing every term of every program in
our dataset. The vectors from a random 30\% of the program pairs are set aside for testing and
the other 70\% are used for training. Within the training data, we use 3-fold
cross-validation to select the best parameters for our models, before training
using those parameters on the full training set and then scoring the models
using the testing set. Each trained model takes in any vector representing a term
in a buggy program, and returns a confidence score representing how likely it is that
that term was one of the terms changed between the fixed and buggy versions.
Thus we can treat the model as providing a \emph{ranking} over all terms by
confidence, where the top result is the one that the model deems most likely
to be changed. For
a given k, we score the model based on Top-k accuracy: for what proportion of the
programs is a correct answer (i.e. a term that is actually changed) present in the
top k results. We retrain the models for each value of k, since a model optimized
to produce the single best result might be different from one optimized to get a
good result into the top three.

In each case the most effective model we trained was a decision tree. (The other
models we tried were Logistic Regression and MLP. Yao-Yuan - anything to add to
this section? in particular, is it actually accurate to say we're doing
cross-validation - the 3-fold thing we do is actually just to select parameters,
and then there's no cross-validation on the final model, right?)

\section{Evaluation}
\label{sec-eval}

We conducted a large-scale empirical evaluation of our algorithm with the
aim of addressing a number of research questions:
\begin{enumerate}

\item[RQ1]{Can we accurately localize non-trivial faults in Python
programs?}

\item[RQ2]{Which features are the most important for Python fault
localization overall?}

\item[RQ3]{Which features are the most important for various categories of
Python defects?}

\item[RQ4]{Is our algorithm accurate on heterogeneous sets of programs?}

\item[RQ5]{FIXME clustering?}

\end{enumerate}

\subsection{Data Set}

Our raw data consist of every Python 3 program that a user executed on
PythonTutor.com~\cite{Guo2013-vu} (not in ``live'' mode) during 2017, other
than those with syntax errors or undefined variables.  Each program which
crashes (throws an uncaught Python exception) is paired with the next
program (by the same user) that does not crash, under the assumption that
the latter is the fixed version of the former. We discard pairs where the
difference between crashing and fixed versions is too high (more than a
stddev above average), since these are most likely to be violations of that
assumption (i.e., the program that does not crash is unrelated to the
crashing program).

In our evaluation, we used FIXME as a training set and FIXME as a held-out
evaluation set. We employed cross-validation~\cite{kohavi} to help address
the potential threat of overfitting.

\subsection{Methodology}

For these evaluations, we employ machine learning algorithm FIXME, which
provided the best balance of FIXME and FIXME (time? accuracy?) on our
training set.

We report Top-1 and Top-3...

We define FIXME-ACCURACY to be FIXME.

\subsection{RQ 1 --- Fault Localization Accuracy}

\begin{figure}
\begin{tikzpicture}
\begin{axis}[
    ybar,
    symbolic x coords={Baseline,Top-1,Top-3},
    xtick=data,
    ymin=0,
    ylabel=Model Accuracy,
    enlarge x limits=0.5,
    legend style={at={(0.5,-0.15)},anchor=north}
]
    \addplot table[x=scoreName, y=score, col sep=comma]{fault-localization.csv};
\end{axis}
\end{tikzpicture}
\caption{Baseline is the normal Python interpreter. Other bars are from decision
trees on the largest dataset (272534 pairs). TODO: also train for Top-2}
\label{fig:full-dataset-acc}
\end{figure}

\begin{figure}
\begin{tikzpicture}
\begin{axis}[
    ybar,
    symbolic x coords={noContext, noTypes, all},
    xtick=data,
    ymin=0,
    ylabel=Model Accuracy,
    enlarge x limits=0.5,
    legend style={at={(0.5,-0.15)},anchor=north}
]
    \addplot table[x=scoreName, y=score, col sep=comma]{removing-features-2.csv};
\end{axis}
\end{tikzpicture}
\caption{Accuracy when features are removed, based on decision trees on a random subset of 20000 pairs.
TODO: Top-2}
\label{fig:removing-features}
\end{figure}


prereq: define accuracy

Goal: Establish that our algorithm works. This is the most basic question,
but also the most relevant. If the reader takes away nothing else, they
should conclude that our algorithm is effective.

PLOT: See Figs. \ref{fig:full-dataset-acc} and \ref{fig:removing-features}

\subsection{RQ 2 --- Feature Predictive Power}

\begin{table}[]
\begin{tabular}{lll}
Name & Category & Gini Importance \\ \bottomrule
Size-P & Contextual (Syntactic)               & 0.112 \\
Stmt-Constr-C3= & Contextual (Syntactic)      & 0.061 \\
Expr-Constr-P=List & Contextual (Syntactic)   & 0.055 \\
PythonBlame & Dynamic (Error location)        & 0.039 \\
Type=unknown & Dynamic (Type)                 & 0.037 \\
Type-P=unknown & Contextual (Type)            & 0.037 \\
PythonMsg=IndexError & Dynamic (Error message)& 0.033 \\
Size & Syntactic                              & 0.028 \\
Expr-Constr-P=Dictionary & Contextual (Syntactic) & 0.027 \\
Size-C1 & Contextual (Syntactic)              & 0.024 \\
Expr-Constr=Var & Syntactic                   & 0.020 \\
Type-C1=unknown & Contextual (Type)           & 0.020 \\
\toprule
\end{tabular}
\caption{Feature predictive power (for a Top-3 Decision Tree
learned on the entire dataset).
TODO: add
better-known measure like ANOVA or ReliefF. TODO: base on Top-1 instead?}
\label{tab-feature-predictive-power}
\end{table}

\tabref{feature-predictive-power} summarizes the relative importance
of the top features in our model. The features are ranked by their
Gini importance (or mean decrease in impurity), a common measure
for decision tree and random forest models~\cite{FIXME}. Informally, the
Gini importance conveys a weighted count of the number of times a feature
is used to split a node: a feature that is learned to guide more model
classification decisions is more important.

prereq: define predictive power (leave-one-out, leave-one-in, Relief-F,
ANOVA, whatever)

PLOT: See Table \ref{tab:feature-predictive-power}

Use as a rough guide: Table 3 on Page 8 of
https://web.eecs.umich.edu/~weimerw/p/weimer-icsm2010.pdf

Goal: Establish that we are smart for including all of these features. Note
which individual features or feature categories were (not) included in
previous work.

Give a simple narrative about our effectiveness: implicitly, we were smart
for deciding to include these features, better ingredients make better
results, we have great results.

TODO: what about turning on/off slicing?

TODO: what about turning on/off types? (is that separate from slicing?)

\subsection{RQ 3 --- Defect Categories}

\begin{figure}
\begin{tikzpicture}
\begin{axis}[
    ybar,
    symbolic x coords={All, TypeError, AttributeError},
    xtick=data,
    ymin=0,
    ylabel=Model Accuracy,
    enlarge x limits=0.5,
    legend style={at={(0.5,-0.15)},anchor=north}
]
    \addplot table[x=scoreName, y=score, col sep=comma]{defect-categories-2.csv};
\end{axis}
\end{tikzpicture}
\caption{Accuracy when the model is trained and tested on only the data exhibiting
the error on the x-axis.}
\label{fig:defect-categories}
\end{figure}

prereq: define defect category. This could be either the raw python
exception name or it could be clusters that we have manually created.
Ranjit notes: result could be ``we need everything for everything'', at
which point this is useless and should be skipped. Per Ben's exam: may find
that for different categories of errors we should use different ML
classifiers.

PLOT Type: Bar Graph (three bars per point on the X axis)

PLOT Y Axis: Leave-One-In Accuracy (normalized per X axis point)
Ranjit notes: on smaller subsets we could rebuild the classifier each time.
Wes notes that if we do, it takes us longer. If we re-use the monolithic
model we can do this very quickly.

PLOT X AXis: Defect Categories

PLOT Bars: Static, Dynamic, Context

Goal: Demonstrate that each category of feature is essential for a certain
piece of the problem. We need ABC feature to handle DEF class of defects,
but ABC feature does not work well on GHI class of defects, for those we
need JKL features. Implicit: we were smart for including all of these
categories.

\subsection{RQ 4 --- Diversity of Programs}

In Nate there were 20 different functions. All the programs were doing the
same 20 functions.

Here, people are doing anything they want. Random stuff going on over here.

We want to show that we get high accuracy *over a heterogenous set of
programs*. Leon?

prereq: cluster the programs based on similarity. Using agglomerative
clustering plus python edit distance. How good the performance will be?
right now it creates a clustering tree. binary tree, based on min distance.
can give it a threshhold for distance (elements w/in a group). don't forget
to rename all variables to X, perhaps normalize (edit distance size).

leon: could also cluster by inconsistency coefficient

PLOT Type: line graph (standard plot)

PLOT X AXIS: clustering distance parameter

PLOT Y AXIS: number of different clusters found

PLOT LINES: line 1 is ``Nate programs'', line 2 is ``25\%, at random, of our dataset''

Goal: show that ``no matter how you decide program differences'', our
dataset is always more diverse/heterogeneous than is previous work. So our
success is amazing.

% \begin{figure}
% \foreach \method in {single, complete, average}
% {
% \begin{tikzpicture}
%   \begin{axis}[
%       xlabel=Threshold,
%       ylabel=\# of clusters,
%     %   ytick distance=2,
%       title=Linkage: \method
%     ]
%     \addplot+[mark=none] table[x=threshold, y=python_twenty_cluster_counts_\method, col sep=comma]{cluster_counts.csv};
% \end{axis}
% \end{tikzpicture}
% }
% \caption{The number of clusters when clustering a baseline of twenty dissimilar programs from
%   Rosetta Code. We give no plot for the baseline of 5000 identical programs, as they were
%   consistently placed in just 1 cluster.}
% \label{fig:diversity-baseline}
% \end{figure}

% \begin{figure}
% \foreach \method in {single, complete, average}
% {
% \begin{tikzpicture}
%   \begin{axis}[
%       xlabel=Threshold,
%       ylabel=\# of clusters,
%     %   ytick distance=1000,
%       title=Linkage: \method
%     ]
%     \foreach \language in {ocaml, python}
%     {
%       \addplot+[mark=none] table[x=threshold, y=\language_cluster_counts_\method, col sep=comma]{cluster_counts.csv};
%     }
% \legend{OCaml, Python}
% \end{axis}
% \end{tikzpicture}
% }
% \caption{Number of clusters when data is clustered by single, complete, and average linkage.
%   A linkage tree is built using agglomerative clustering, and the resulting tree is broken up
%   into clusters by thresholding the inconsistency coefficient. The inconsistency coefficient
%   of a link is its distance, z-scored against other nearby links.}
% \label{fig:diversity}
% \end{figure}

\begin{figure}
\begin{tikzpicture}
  \begin{axis}[
      xlabel=Clustering threshold parameter,
      ylabel=\# of naturally occuring program clusters,
      xmax=1.25,
      legend style={at={(0.5,0.25)},anchor=north}
    %   ytick distance=1000,
    %   title=Linkage: single
    ]
    \foreach \language in {ocaml, python}
    {
      \addplot+[mark=none] table[x=threshold, y=\language_cluster_counts_single, col sep=comma]{cluster_counts.csv};
    }
\legend{NATE OCaml data, Our Python data}
\end{axis}
\end{tikzpicture}
\caption{Number of clusters when data is clustered by single linkage.
  A linkage tree is built using agglomerative clustering, and the resulting tree is broken up
  into clusters by thresholding the inconsistency coefficient. The inconsistency coefficient
  of a link is its distance, z-scored against other nearby links. TODO: find a way to include
  the Rosetta Code baseline in this graph? better not to devote multiple graphs to this issue.}
\label{fig:diversity}
\end{figure}

\subsection{RQ 5 --- Wishlist: Partitioning Kinds of Errors}

Can we partition the kinds of errors that people make? What are common
misconceptions? Do these correspond to paths of decision trees.

If we don't get to the source-sink thing, it could go here.

I don't think we have time, in this paper, to actually implement a
two-level machine learning.

\subsection{Qualitative Analysis}

Pick out one example where we did poorly. Walk through why it is ``out of
scope'', in some sense: we would need features that capture XYZ to handle
it correctly, but those are either expensive or undecideable. Hint at
future work.

Pick out a few examples where we did well and the baseline did poorly. Walk
through how interesting and indicative they are and explain why we did well.

\subsection{Evaluation Summary}

Summarize the evaluation. One sentence per subsection.

\subsection{Threats to validity}

\emph{Overfitting and other machine learning issues} Yao-Yuan, how do we avoid
these?

\emph{Language choice} We only showed that this technique works for Python 3.
However, \cite{learning-to-blame} uses a similar technique to great effect in
OCaml, and if it works on these two quite-different languages then we believe
it would work on others as well. Additionally, Python is a common choice of
language for introductory programming courses, so even a technique that worked
only on Python would still be very useful for helping novices.

\emph{Target population} We don't know exactly who our study population is -
who wrote the programs in the dataset and what exactly they were trying to write.
It is likely that they are primarily students and not professional software engineers,
and thus possible that our technique does not generalize well to the kinds of
bugs that would be found in an industrial setting rather than an educational one.

\emph{Automatic labeling} Since our dataset is so large, we were forced to use an automatic
scheme to label bugs rather than marking them by hand. We chose to use the diffs
between each program that crashes and the next program that does not, but this method
is not 100\% accurate because the next program may contain additional changes
beyond what was needed to fix the bug, or it may indeed be an entirely unrelated
program. We mitigate this problem by discarding as outliers program pairs that have
changed too much. We also note that if our `true' accuracy is indeed a little
smaller than our reported accuracy because of this effect, then the `true' baseline
is also lower than the reported baseline for the same reason, so our
\emph{margin} of effectiveness is unlikely to change much.

\section{Related Work}
FIXME: This must all be rewritten or dropped, as it is directly copied from
my previous papers. It serves as a placeholder to indicate length and to
remind us of some Software Engineering bits we might cite.

There are several fault localization techniques that rely much more heavily
on dynamic information\footnote{Note: we do not directly compare to these
state-of-the-art fault localization techniques for several reasons.  First,
the benchmarks used in this paper do not always have available the test
suites needed by dynamic techniques and, conversely, those used in the
previous work do not have the publicly available bug reports required
by our technique.  Additionally, we return an answer only when confident in
our result set and also employ a top list of all available results, while
previous work is typically evaluated using a score metric.  We consider the
techniques complimentary.}~\cite
{harrold05,Renieris03,cleve05,wang09}.
In general,
these techniques leverage differences between passing and failing program
executions.  While effective, this type of approach requires program traces
for not only the specific fault at hand but also from a comprehensive regression
test suite for comparison.
A hybrid approach that considers both dynamic
traces and static natural language similarity was recently proposed by
Medini \textit{et al.}~\cite{Medini11}.
In contrast to these approaches, our technique relies purely on static
information that is readily available in most commercial systems with minimal
additional developer effort.

Prabhakararao and Ruthruff \emph{et al.} performed two human studies
to gauge the effectiveness of an interactive fault localization tool
developed for end users with little to no
experience~\cite{Prabhakararao03,ruthruff05}.  The goal of their
studies was to evaluate the use of feedback when locating faults and
to generally study the process of fault localization, especially by
users with no expert domain knowledge of the source.  By comparison,
our human study also examines the fault localization process but for
the purpose of evaluating software quality metrics.  We are less
interested in the specific process and more concerned with the
resulting accuracy and the human intuitions about the code in
question.  Additionally, our human study is of a much broader scope and
thus we hope it is more generalizable.
%Additionally, their studies included 10 and 20
%participants respectively using a spreadsheet-based visualization tool.
%Our study of 61 humans looking at actual source code potentially provides
%more generalizable results and targets specifically the area of software
%maintenance completed by trained programmers.


Ashok \textit{et al.} propose a similar natural language search technique
in which users can match an incoming report to previous reports,
programmers and source code~\cite{Ashok09}.  By comparison, our technique
is more lightweight and focuses only on searching the code and the
defect report.

Jones \textit{et al.} developed Tarantula, a technique that performs
fault localization based on the insight that statements executed often
during failed test cases likely account for potential fault
locations~\cite{harrold05}. Their approach is quite effective when
a rich, indicative test suite is available and can be run as part of
the fault localization process. It thus requires the fault-inducing
input but not any natural language defect report. By contrast,
our approach is lightweight, does not require an indicative test
suite or fault-inducing input, but does require a natural language
defect report. Both approaches will yield comparable performance, and
could even be used in tandem.

Cleve and Zeller localize faults by finding differences between
correct and failing program execution states, limiting the scope of
their search to only variables and values of interest to the fault in
question~\cite{cleve05}. Notably, they focus on those variable and
values that are relevant to the failure and to those program execution
points where transitions occur and those variables become causes of
failure. Their approach is in a strong sense finer-grained than ours:
while nothing prevents our technique from being applied at the level
of methods instead of files, their technique can give very precise
information such as ``the transition to failure happened when $x$
became 2.'' Our approach is lighter-weight and does not require
that the program be run, but it does require defect reports.

Renieris and Rice use a ``nearest neighbor'' technique in their
Whither tool to identify faults based on exposing differences in
faulty and non-faulty runs that take very similar executions
paths~\cite{Renieris03}. They assume a large number of correct runs
(e.g., normal test cases) and one failing run. Their approach uses a
distance criterion to select the correct run that is closest to the
failing run and produces a report of ``suspicious'' parts of the
program. By comparison, we chose to limit the programmatic information
used by our technique to only that which was reported by users: we
do not use test case runs but do need natural language.

Liblit \textit{et al.} use Cooperative Bug Isolation, a statistical
approach to isolate multiple bugs within a program given a deployed user
base. By analyzing large amounts of collected execution data from real
users, they can successfully differentiate between different causes of
faults in failing software~\cite{liblit05}. Their technique produces
a ranked list of very specific fault localizations (e.g., ``the fault
occurs when $i > arrayLen$ on line 57''). In general, their technique
can produce more precise results than ours, but it requires a set of
deployed users and works best on those bugs experienced by many users.
By contrast, we do not require that the program be runnable, much less
deployed, and use only natural language defect report text.

\section{Conclusion}

\bibliographystyle{abbrv}
\bibliography{nanomaly,sw,temp,slice}

\end{document}
